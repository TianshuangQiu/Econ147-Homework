\documentclass[12pt]{article}
\usepackage[usenames]{color} %used for font color
\usepackage{amsmath, amssymb, amsthm}
\usepackage{wasysym}
\usepackage[utf8]{inputenc} %useful to type directly diacritic characters
\usepackage{graphicx}
\usepackage{caption}
\usepackage{subcaption}
\usepackage{float}
\usepackage{mathtools}
\usepackage [english]{babel}
\usepackage [autostyle, english = american]{csquotes}
\MakeOuterQuote{"}
\graphicspath{ {./} }
\newcommand{\Z}{\mathbb{Z}}
\newcommand{\N}{\mathbb{N}}
\newcommand{\R}{\mathbb{R}}
\newcommand{\Q}{\mathbb{Q}}
\newcommand{\prob}{\mathbb{P}}
\newcommand{\degrees}{^{\circ}}
\DeclarePairedDelimiter\ceil{\lceil}{\rceil}
\DeclarePairedDelimiter\floor{\lfloor}{\rfloor}

\author{Tianshuang (Ethan) Qiu}
\begin{document}
\title{Algorithmic Economics, PS3}
\maketitle

\section*{Problem 1}
The shaply value is defined as $\phi_i(v) = \frac{1}{n}\sum_R[v(P_i^R \cup {i})-v(P_i^R)]$
where $P_i^R$ is the set of players in the coalition $R$ that precede $i$ in the order $R$.

Consider two games $v. w$, let $\Delta_i^v(S) \leq \Delta_i^w(S)$ for all $S \subseteq N$. Then we must have 
that $\phi_i^v \leq \phi_i^w$ since each element of the average is smaller for $v$ than for $w$.
Thus we have shown the the Shapley value satisfies the axiom of marginality.

Now consider the axiom of subsitute player. Consider a game $v$ and players $i, j \in N$ be subsitutes. Then we have 
that $\Delta_i^v(S) = \Delta_i^w(S)$ for any $i, j \not\in S$. Then when we compute the Shapley value, 
for each term before $i$ and $j$ are added, the difference in value after they get added is the same. 
Therefore, the Shapley value satisfies the axiom of substitute player.

\newpage
\section*{2}
Let $(N. v_T)$ be a simple game, let $T \subseteq N$, and $v=a\times v_T$, $a > 0$.
Consider $i \not \in T$, then the agent's inclusion is irrelevant to the overall value of the coalition.
Therefore, no matter the order of the agent in the coalition, the value itself does not change, therefore 
$\Delta_i^v(S) = 0$.

Now consider $i \in T$, then the agent's inclusion adds $a$ to the value of the coalition only if it is the last member of $T$. Therefore 
for a given $i \in T$, there exists $\frac{N!}{T}$ permutations where $i$ is the last member of $T$. Consider if we pick out all members of $T$ from all 
permutations, then this forms a random permutation of $T$. Any given random permutation of $T$ is equally likely to be sampled from $N!$, then since we limit 
ourselves to $i$ being the last element, we have a $\frac{1}{|T|}$ chance of sampling the correct permutation. Therefore, we have a total amount of 
$\frac{N!}{|T|}$ such permutations. Therefore, we have $\phi_i(v) = \frac{1}{N!}\frac{N!}{|T|}a = \frac{a}{|T|}$.


\newpage
\section*{3}
Let $s$ satisfy marginality. Let $\Delta_i^v(P) = \Delta_i^w(P)$ for all $P \subseteq N$. Then we have $\Delta_i^v(P) \leq \Delta_i^w(P)$, 
by marginality we have $s_i^v \leq s_i^w$. Similarly, we have $\Delta_i^w(P) \leq \Delta_i^v(P)$, and by marginality we have $s_i^w \leq s_i^v$. 
Therefore we can conclude that $s_i^v = s_i^w$. 

Thus $s$ must depend only on the palyer's marginal contributions. If it depends on any other factor, 
we can construct a game $v$ with agents $i, j$ such that $\Delta_i^v(P) = \Delta_i^w(P)$ for all $P \subseteq N$, but they have different values on the factor 
that $s$ depends on. Then we would have $s_i^v \neq s_i^w$, which is absurd. Therefore $s$ must only be a function of the player's marginal contributions.


\newpage
\section*{4}
\subsection*{1}
If the seller is not present in the coalition, no value can be generated. Therefore if 
$1 \not\in S$, we have $v(S)=0$. If the seller is alone, the item is worth $w_1$ to him. If a buyer is present, 
the seller sells the item to the buyer, and the value of the coalition is the value of the buyer. When both buyers 
are present, the seller sells the item to the buyer with the highest value ($w_3 > w_2 > w_1$). Therefore, the value of the grand coalition is 
$w_3$.

\subsection*{2}
Let $X = (x_1, x_2, x_3)$ be the core of the game. We know that $V(X)=w_3$, and to allocate this value, 
we know that $1$ sold to $3$, so $x_2=0$. Since if $x_2 > 0$, $V(x_1, x_3) = w_3$, which is greater than their 
current value $w_3-x_2$. Therefore, the core of the game has $x_2=0$ 

Consider $x_1$, since $V(N) = w_3$, $x_1 \leq w_3$. If $x_1 < w_2$, then $V(\{1, 2\}) = w_2$, which is greater than 
$V(\{1, 2\}) < w_2 + 0 = w_2$, and thus $\{1, 2\}$ can get a better value by excluding $3$. Therefore, to maintain the core, we must have 
$w_3 \geq x_1 \geq w_2$

Consider $x_3$, if $x_3> w_3-w_2$, then $x_1 = w_3-x_3 < w_2$. $V(\{1, 2\}) = w_2$, which is greater than 
$V(\{1, 2\}) < w_2 + 0 = w_2$, and thus $\{1, 2\}$ can get a better value by excluding $3$. Therefore, to maintain the core, we must have
$w_3-w_2 \geq x_3 \geq 0$
% TODO: finish the proof 

From the viewpoint of the seller, the best imputation of maximizes $x_1$, so $x_1 = w_3$ and $x_2 = x_3 = 0$. 
From the viewpoint of the buyers, $x_2=0$ regardless of the coalition, and $x_3$ has a maximum value of $w_3-w_2$, so 
the best imputation is $x_1 = w_2$, $x_2 = 0$, and $x_3 = w_3-w_2$.


\subsection*{3}
For each of the 6 potential orders, we compute the Shapley value for each.

We have $\phi_1(v) = (2w_1+w_2+3w_3)/6$, $\phi_2(v) = (w_2-w_1)/6$,
$\phi_3(v) = (3w_3-w_1-2w_2)/6$

Since $x_2 > 0$, the Shapley value is not in the core.


\newpage
\section*{5}
\subsection*{1}
Veto players are a set of players that are in \textbf{all} winning sets. Now let $V = \emptyset$
Let $x$ be an imputation, select $S=\{i \in N x_i>0\}$, choose $i \in S$ and a 
coalition T such that $i \not\in T$. 
$\sum_{j \in T}x_j \leq 1-x_i < 1 = V(T)$. We know that $T$ exists because $V(N)$, and since the set of 
veto players is empty, we can find such a coalition. Therefore, we have found another coalition that gets the same value 
by excluding $i$. Therefore, the core of the game is empty.

\subsection*{2}
Let $V \neq \emptyset$, and let $x$ be a core imputation. Assume that $\exists j$ such that $x_j > 0$ but $x_j \not \in V$. Then we can construct 
a coalition of only veto players, and $\sum_{j\in V}x_i \leq 1-x_j < 1 = V(V)$. Thus the veto players can get a higher value 
if they exclude $j$. This violates the core. Therefore, we must have $x_j = 0$ for all $j \not\in V$.

Now consider the converse. Let $x$ be an imputation, and $x_j = 0$ for all $j \not\in V$. Then suppose that $x$ is not the core, so there exists and coalition $Q$ such 
that $V(Q) > V(S)$. However, since we know that $V(N) = 1$, $V(P) \leq 1 \forall P \subseteq N$. Thus it is impossible that 
$V(Q) > V(S) = 1$. Then we consider if we can have less players in our set. However, by the definition of veto players, 
we know that $V(S) = 1$ for all $S \subseteq T$, thus if $Q$ excludes any veto player, $V(Q) = 0$.
Therefore, such a $Q$ does not exist, and $x$ is in the core.


\newpage
\section*{6}

\subsection*{1}
$\Delta_i^v(S(\geq, i)) = V(S\cup\{i\}) - V(S)$ for some ordering of $S$. Since we know that $i \not\in S$, $A\cap B = \emptyset$
By convexity we know that $V(S\cup\{i\}) \geq V(S) + V(\{i\})$. Thus $\Delta_i^v(S(\geq, i)) = V(S\cup\{i\}) - V(S) \geq V(\{i\}) \geq 0$.
Thus $x_i \geq 0$

Consider $\sum_{i \in N}x_i$, we know that the grand coalition has value $V(N)$, and $V(\emptyset) = 0$. Thus in our ordering, 
we can compute $\sum_{i \in N}\Delta_i^v S(\geq, i) = V(\{1\}) - V(\emptyset)+V(\{1,2\}) - V(\{1\}) + \ldots$. 
This sum telescopes to $V(N) - V(\emptyset) = V(N)$. Thus $\sum_{i \in N}x_i = V(N)$
\subsection*{2}
Let $A\subseteq B$, and let $C = A \setminus B$. 
We first apply the definition of $\Delta$
\[\Delta_i^v(A) = V(A\cup\{i\}) - V(A)\]
\[\Delta_i^v(B) = V(B\cup\{i\}) - V(B) = V(C\cup (A \cup\{i\})) - V(A \cup C)\]
Then we apply the property of convexity
\[\Delta_i^v(B) \geq V(C)+V(A \cup\{i\}) - V(A\cup C)\]
\[\Delta_i^v(A) = V(A\cup\{i\}) - V(A) + V(C) - V(C)\]
Finally, we do some algebraic manipulation
\[\Delta_i^v(A)-V(C) \leq V(A\cup\{i\}) - V(A \cup C)\]
\[\Delta_i^v(B)-V(C) \geq V(A \cup \{i\}) - V(A\cup C)\]
Thus we have $\Delta_i^v(A)-V(C) \leq \Delta_i^v(B)-V(C)$, and thus $\Delta_i^v(A) \leq \Delta_i^v(B)$

\subsection*{3}
In part 1, we have proven that $x_i \geq 0$ and $\sum_i x_i = V(N)$, thus $x$ is an imputation.
Let $x$ be an imputation, for any subset of agents, we can fix an ordering $\{x_j \ldots x_k\}$, we have $V(\{x_j \ldots x_k\}) = \sum_{i=j}^k \Delta x_i$

Using the same reasoning as subsection 1, we can express $V(S)$ as a telescoping sum. However, when $S$ was in the original superset, 
there are $j-1$ elements before the first element of $S$, and $n-k$ elements after the last element of $S$. Thus by convexity, the marginal 
contributions of $S$ would have been larger when it was part of the superset compared to when it was part of the subset. 

Thus the agents in this set do not benefit from deviating.
\[V(S) = \sum_{i = j}^{k} \Delta^v(\{x_j \ldots x_k\}) \leq \sum_{i \in S}x_i\]


\subsection*{4}
We know that the from the previous subsection, $\Delta_i^v S(\geq, i)$ is in the core for any ordering. Since the core is 
a convex set, the average of all $\Delta_i^v S(\geq, i)$ is also in the core. This value is precisely the Shapley value. 
Therefore the Shapley value is in the core.


\newpage
\section*{7}
We can show that the core of a simple game is non-empty by proving a
certain imputation lies in the core. Specifically the imputation where each
agent $x \in T$ gets $\frac{1}{|T|}$. 

For all players, $V(x_i) \geq 0$. 

Let our assignment be $X$. For all coalitions, if $T \subseteq S$, then $V(S) = \sum_{i \in S} x_i = 1 = V(N)$. 
In this case, the players do not get a benefit from leaving the coalition. If $T \not\subseteq S$, 
then $V(S) = 0 < \sum_{i \in S} x_i = 0 < \sum_{i \in S}x_i = \frac{n}{|T|}$ where $n$ is the amount of $T$ members in $S$.
Thus in this case, the players also do not get a benefit from leaving the coalition. Finally $\sum_{i \in S}x_i = 1 = V(N)$. 
Thus $X$ is in the core.

To describe all core imputations, we must have $x_i = 0$ for $i \not \in T$. Otherwise, the coalition 
can exclude the player not in the winning section and get a better value. For members of the $T$ coalition, we simply need to 
ensure that $\sum_{i \in T} x_i= 1 = V(N)$ and $x_i \geq 0$. This ensures that we satisfy the requirements of 
an imputation, and the $T$ players cannot benefit by forming another coalition. 

\newpage
\section*{8}
\subsection*{1}
Let the core of the game be non-empty. So let $x$ be in the core. By definition, $V(S) \geq \sum_{i\in S} x_i$ for any $S \subseteq N$. 
Furthermore, we have $V(S)= f(|S|)$. Consider a function that re-orders the elements, let the allocation be $x^y = \{x_{y1}, x_{y2}, \ldots, x_{yn}\}$
We will show that for every mapping $y$, $x^y$ is in the core. 

Let the remapped set be $S* = \{y_1, y_2, \ldots\}$
Let the original set be $S'$. Since $x$ is in the core, we know that $V(S') \leq \sum_{j=0}^kx_{i+j}$. Since it was 
just relabelling, we know that $V(S) = V(S') = f(|S|)$ since the total amount of items within this set has not changed.

Now given all shuffling of $x$, $x^y$ also lies in the core. We know that the average must also lie in the core since 
the core is convex. The average would be $\frac{\sum_{i=1}^{n}x_i}{n}$ which by definition is equal to $\frac{V(N)}{n} = \frac{f(n)}{n}$.

Thus (f(n)/n, f(n)/n, \ldots, f(n)/n) is in the core.


\subsection*{2}
If $\frac{f(x)}{x}\leq \frac{f(n)}{n}$ for all $x \in \{1, \ldots, n\}$, then the core is 
non-empty. 

$V(S) = f(|S|) \leq \frac{f(n)|S|}{n}$
Let $x$ be an imputation that assigns $\frac{f(n)}{n}$ to all agents. Since everyone in this imputation gets 
the same assignment, $\sum_{i \in S}x_i = |S| \frac{f(n)}{n}$

Thus $V(S) = f(S) \leq \frac{|S|f(n)}{n} = \sum_{i \in S}x_i$ for all $S$, and $x$ is in the core. 
Thus the core is not empty.

\subsection*{3}
Assume that $n \geq 2$, and we want to show that the core is empty. Assume that there exists an imputation $x$ that is in the core. 
Select some $j \in N$, by the problem statement
$\sum_{i \neq j}x_i = (M-n)-x_j < M-1$. We know that $x_j \geq 0$ and $n \geq 2$, thus 
$\sum_{i \neq j}x_i = V(\{i \in N: i \neq j\})$

Since $V(S) \geq \sum_{i \in S}x_i$, all the members but one will always have an incentive to devaite and 
throw all trash into the excluded player's garden. Thus the core is empty.
\end{document}