\documentclass[12pt]{article}
\usepackage[usenames]{color} %used for font color
\usepackage{amsmath, amssymb, amsthm}
\usepackage{wasysym}
\usepackage[utf8]{inputenc} %useful to type directly diacritic characters
\usepackage{graphicx}
\usepackage{caption}
\usepackage{subcaption}
\usepackage{float}
\usepackage{mathtools}
\usepackage [english]{babel}
\usepackage [autostyle, english = american]{csquotes}
\MakeOuterQuote{"}
\graphicspath{ {./} }
\newcommand{\Z}{\mathbb{Z}}
\newcommand{\N}{\mathbb{N}}
\newcommand{\R}{\mathbb{R}}
\newcommand{\Q}{\mathbb{Q}}
\newcommand{\prob}{\mathbb{P}}
\newcommand{\degrees}{^{\circ}}
\DeclarePairedDelimiter\ceil{\lceil}{\rceil}
\DeclarePairedDelimiter\floor{\lfloor}{\rfloor}

\author{Tianshuang (Ethan) Qiu}
\begin{document}
\title{Algorithmic Economics, PS2}
\maketitle

\section*{Problem 1}
\subsection*{1}
\paragraph*{Forward}
Let $\exists p \in \Delta(X)$ with $\mu(A) = \sum_{x\in A}p(x)$
for all $A \in X$. Since $p$ is a lottery, $p(x_i)\geq 0$, thus 
$\mu(A)\sum_{x\in A}p(x) \geq 0$. Since $\sum_{x\in X} p(x)=1$, then 
$\mu(X) = 1$. For disjoint sets $A, B \subset X$, we can find $p$ such that 
$\mu(A \cup B) = \sum_{x \in A}p(x) +\sum_{x \in B}p(x) = \mu(A) + \mu(B)$ due 
to disjoint sets (elements are distinct within A, B). Thus we have shown all 
properties of probability measures and $\mu$ is a PM.

\paragraph*{Backward}
Now let $\mu$ be a probability measure, then for any 
$A = \{x_1, x_2,$ $ \ldots, x_n\}  \subseteq X$, 
we construct a lottery $p = (\mu(x_1), \mu(x_2), \ldots,$ $ \mu(x_n), 1-\mu(A), 0, \ldots)$.
By our construction we have $p(X)=1$, and since $\mu$ is a probability measure, $p(x_i)\geq 0$
These satisfy the properties of the lottery, so we have found $p \in \Delta(x)$ with 
$\mu(A)=\sum_{x\in A}p(x)$

$\blacksquare$
\newpage



\subsection*{2}
Let $v \in \R^d$, then $u_v(p) = \sum_{x\in X} v_x p_x \in \R$. 
Let $p, q \in \Delta$, by our previous definition $u_v(p) = v \cdot p$, which is an element in 
the reals. By orderedness of the real numbers, any $r_1, r_2 \in \R$ satisfies $r_1 \geq r_2$ or 
$r_2 \geq r_1$, thus $p \succeq q$ or $q \succeq p$, and it is complete.

Following our previous conclusion, since the function $u_v$ maps to real numbers, 
which satisfy transitivity by ordered field theorem. $p \succeq q, q \succeq r \to v \cdot p \geq v \cdot q$, $v \cdot q \geq v \cdot r$.
Thus by order of reals we have $v \cdot p \geq v \cdot r$, and $p \succeq r$, thus it is transitive.

$\blacksquare$
\newpage


\subsection*{3}
Let $p, q, r \in \Delta(X), \lambda \in (0, 1)$. For some $v \in \R^d$, 
$p \succeq_v q \to v \cdot p \geq v \cdot q$. 
\[
\lambda p + (1-\lambda)r = \lambda \sum_{x\in X} v_x p_x + (1-\lambda) \sum_{x\in X} v_x r_x
\]
\[
\lambda q + (1-\lambda)r = \lambda \sum_{x\in X} v_x q_x + (1-\lambda) \sum_{x\in X} v_x r_x
\]
Since we are operating in the reals which are well ordered, we can subtract $(1-\lambda) \sum_{x\in X} v_x r_x \in \R$ 
from both equations to see that we are really comparing $\lambda v \cdot p, \lambda v \cdot q$. Since $\lambda \in (0, 1)$, 
we can safely multiply by its inverse and not affect the ordering. Thus have shown that the expression is equivalent to comparing 
$v \cdot p, v \cdot q$, and by definition of $p \succeq_v q$ iff $v \cdot p \geq v \cdot q$

$\blacksquare$
\newpage


\subsection*{4}
I would prefer $p_A$, and I would prefer $p_B'$

Under $\succeq_v$, we can compute the expected utility of all values. Let $v = [v_1, v_2, v_3]$
\[u_v(p_A) = v_2\]
\[u_v(p_B) = 0.01v_1 + 0.89v_2 + 0.1 v_3\]
\[u_v(p_{A'}) = 0.89v_1 + 0.11v_2\]
\[u_v(p_{B'}) = 0.9v_1+0.1 v_3\]

By our preference relation we have $p_A \succeq p_B$ and $p_{B'} \succeq p_{A'}$. 
By using the equivalent definition over $\geq$, we can see that $u_v(p_{B'}) \geq u_v(p_{A'})$, 
which implies that $0.01v_1 + 0.1v_3 - 0.11v_2 \geq 0$. Then by $p_A \succeq p_B$ we have 
$0.11v_2-0.01v_1-0.1v_3 \geq 0$, then we can rearrange to get
\[0.11v_2 \leq 0.01v_1 + 0.1v_3\]
\[0.11v_2 \geq 0.01v_1 + 0.1v_3\]
If we are only considering strict preferences, then we would remove the equal sign and there would 
be no solution to this system of equations under real vectors.
Thus it cannot be consistent with any preference relation.

$\blacksquare$
\newpage

\subsection*{5}
\paragraph*{Nonempty}
Consider an arbitrary preference over $\Delta(X)$ that ranks the degenerate lotteries as $[x_1, x_2, \ldots, x_n]$ where 
$n = |X|$ and $x_1$ is the least preferred while $x_n$ is the most preferred. We can then construct $v = [1, 2, \ldots, n]$.
Then let $x_m, x_k$ be elements of $X$ and WLOG let $m \geq k$. Then our preference $\succeq_v$ computes $v \cdot \mathbf{1}_m = m$ 
and $v \cdot \mathbf{1}_k = k$. Since $m \geq k, m^2 \geq k^2$, thus we have found one such $v$ and the set must be nonempty.

\paragraph{Implication}
Let $\succeq_v \succeq_w \in L(\geq)$, let $p,q \in X$ such that $p \succeq_v q$ and 
$p \succeq_w q$. Per definition of $\succeq_v$, we have $v \cdot p \geq v \cdot q$, and 
$w \cdot p \geq w \cdot q$. 
Let $u = \alpha v + \beta w$, then 
\[u \cdot p = \sum_{x \in X} u_x p_x = \sum_{x \in X} \alpha v_x p_x + \beta w_x p_x \]
\[u \cdot q = \sum_{x \in X} u_x q_x = \sum_{x \in X} \alpha v_x q_x + \beta w_x q_x \]

By $p \succeq_v q$ and  $p \succeq_w q$ we have $\sum_{x \in X} v_x p_x \geq \sum_{x \in X} v_x q_x$, 
and $\sum_{x \in X} w_x p_x \geq \sum_{x \in X} w_x q_x$, thus $u \cdot p \geq u \cdot q$, and 
$\succeq_{\alpha v + \beta w} \in L(\geq)$

$\blacksquare$
\newpage


\subsection*{6}
\paragraph*{Example}
Let $|X|={a,b,c}$, and we prefer it in the order of $a \geq b \geq c$. Let $p = [0.8, 0.1, 0.1]$, 
q = $[0.5, 0.5, 0]$. 

Let $v = [1, 0.9, 0.8]$, $v \cdot \mathbf{1}_a \geq v \cdot \mathbf{1}_b \geq v \cdot \mathbf{1}_c$, 
so $\succeq_v \in L(\geq)$. $u_v(p) = 0.97 > u_v(q)= 0.95$, so $p \succ_v q$.

Let $w = [1, 0.9, -100]$, $w \cdot \mathbf{1}_a \geq w \cdot \mathbf{1}_b \geq w \cdot \mathbf{1}_c$, 
so $\succeq_v \in L(\geq)$. $u_w(p) = -9.11 < u(q) = 0.95$


\paragraph*{Proof}Suppose $|X|=2$, and by definition of $L(\geq)$, $p \geq q \iff p \succeq q$. 
Then we define degenerate lotteries $a=[1,0], b=[0,1]$. WLOG let us define a 
preference of $a \geq b$. Then we must have $v \cdot a \geq v \cdot b$. Then by definition of $u_v$, 
we have $v_0 \geq v_1$. We can repeat the above reasoning to get $w_0 \geq w_1$
Then since $p \succ_v q$, we have
\[v \cdot p > v \cdot q\]
\[v_0 p_0 + v_1 p_1 > v_0 q_0 + v_1 q_1\]
Since $p, q$ are lotteries, $p_0 + p_1 = 1, q_0+q_1=1$, thus
\[v_0 p_0 + v_1 (1-p_0) > v_0 q_0 + v_1 (1-q_0)\]
\[v_0p_0 + v_1-v_1p_0 > v_0 q_0 + v_1+v_1-v_1q_0\]
\[p_0(v_0-v_1) > q_0(v_0-v_1)\]
\[p_0 > q_0\]

However, since we also have $q \succ_w p$, we repeat the above derivation to get 
\[p_0 < q_0\]
This is a contradiction, and then $|X|>2$. When $|X|=3$, $p_0+p_1+p_2=1$, and 
this gives one more degree of freedom when solving the system of equation, so we 
can successfully find examples. 
\newpage


\subsection*{7}

\paragraph*{Forward}
Let $p, q \in \Delta(X)$ and $p \succeq q$ for all $\succeq \in L(\geq)$. Suppose the $\exists x \in X$ 
such that $p(U(x))<q(U(x))$. Then this means that $\sum_{z \in U(x)}p_z < \sum_{z \in U(x)}q_z$. Then define 
$v = [1 \text{ if $y \in u(X)$}, 0 \text{ otherwise}]$.
Consider arbitrary $a, b \in X$, WLOG let $a \geq b$. 

If $a \geq b \geq x$, $u_v(a)=u_v(b)=u_v(x)=1$.
If $a \geq x \geq b$, then $u_v(a) = u_v(x) = 1 \geq u_v(b) = 0$. If $x \geq a \geq b$, then 
$u_v(x) = 1 \geq u_v(a) = u_v(b) = 0$. 
To make the above preferences strict, consider that we add $\epsilon$ to the least favorite, 
$2 \epsilon$ to the second, and $n \epsilon$ to the favorite. This small addition makes the preference strict.
In any case, we see that $u_v(a) \geq u_v(b)$, thus $\succeq_v \in L(\geq)$


$u_v(p) = v \cdot p$, $u_v(q) = v \cdot q$. By our definition of $v$, since $\epsilon \to 0$,  
$v \cdot p = \sum_{z \in U(x)}p_z <  \sum_{z \in U(x)}q_z = v \cdot p$. 

This is a contradiction to the property that $p \succeq q$ for all $\succeq \in L(\geq)$. Thus such an 
$x$ does not exist and $p(U(x))\geq(U(x)) \forall x \in X$


\paragraph*{Backward}
Let $p(U(x)) \geq q(U(x)) \forall x \in X$. Then assume that $\exists \succeq_v \in L(\geq)$ such that 
$p \prec_v q$

Consider a re-ordering of the vectors such that the least preferred element is the first element and the 
most perferred element last. Let $X = [x_1, x_2, \ldots, x_n]$. Since $p \prec_v q$, $v \cdot p < v \cdot q$.
Since $\succeq_v \in L(\geq)$, $v_i \geq v_j \forall i \geq j$
Thus we can write
\[\sum_{i=j}^n v_i p_i = \sum_{i=1}^n v_1 p_i + \sum_{i=2}^n (v_2-v_1) p_i + \cdots + (v_n-v_{n-1})p_n\]
\[\sum_{i=j}^n v_i q_i = \sum_{i=1}^n v_1 q_i + \sum_{i=2}^n (v_2-v_1) q_i + \cdots + (v_n-v_{n-1})q_n\]
For each element, $v_j\sum_{i=j}^n p_i \geq v_j\sum_{i=j}^n q_i$, thus we conclude that 
$v \cdot p \geq v \cdot q$

This is a contradiction to our assumption that $p \prec_v q$, thus under our assumption, for all 
$\succeq \in L(\geq)$, $p \succeq q$
\newpage

\section*{Problem 2}
\paragraph*{Forward}
Let $v(\emptyset) < v(x) \forall x \in X$. Let $p \geq q$, then by definition $p_x \geq q_x$
Thus $\sum_{x\in X}p_x v(x) \geq \sum_{x\in X}q_x v(x)$, and $(1-\sum_{x\in X}p_x)v(\emptyset) \leq (1-\sum_{x\in X}q_x)v(\emptyset)$

Consider 
\[\sum_{x\in X}p_x v(x) + (1-\sum_{x\in X}p_x)v(\emptyset) - \sum_{x\in X}q_x v(x) - (1-\sum_{x\in X}q_x)v(\emptyset)\]
\[\sum{x \in X} (p_x - q_x) v(x) + (1-\sum_{x\in X}p_x -1+ \sum_{x \in X}q_x)v(\emptyset)\]
\[\sum{x \in X}_(p_x - q_x) v(x) + (-\sum_{x\in X}p_x + \sum_{x \in X}q_x)v(\emptyset)\]
Since $p_x \geq q_x$ and $v(\emptyset) < v(x) \forall x \in X$ the above expression is positive.
Thus 
\[\sum_{x\in X}p_x v(x) + (1-\sum_{x\in X}p_x)v(\emptyset) \geq \sum_{x\in X}q_x v(x) + (1-\sum_{x\in X}q_x)v(\emptyset)\]
$p_x \geq q_x \implies p \succeq_v q$. We can get the strict proof by simply removing 
the equal condition in $\geq$

\paragraph*{Backward}
Let $\succeq_v$ be monotonic. Suppose that there exists $x \in X$ such that $v(\emptyset) \geq v(x)$. Let $v(\emptyset) \geq v(x_i)$ 
Choose $p=[0.5 \text{if $x=x_i$, 0 otherwise}]$, $q=\vec{0}$. We observe that $p > q$, by our assumption $p \succ q$, and
\[\sum_{x\in X}p_x v(x) + (1-\sum_{x\in X}p_x)v(\emptyset) > \sum_{x\in X}q_x v(x) + (1-\sum_{x\in X}q_x)v(\emptyset)\]
However when we compute this expression, we find that LHS = $0.5v(x_i) + 0.5v(\emptyset)$, and RHS = $v(\emptyset)$

By our assumption $v(\emptyset) \geq v(x_i)$, so we have RHS $\geq$ LHS, but by monotonicity we should have 
RHS $<$ LHS.

We have reached a contradiction, so there cannot exist such an $x$, and  $v(\emptyset) < v(x)$

$\blacksquare$

\newpage

\section*{Problem 3}
Consider a cake cutting problem with $n=2$. Agent 0 has utility $U_0$, and is assigned a piece $P_0$, 
similarly agent 1 has utility $U_1$ and piece $P_1$. The utility of the whole cake is defined as 1.

\paragraph*{Forward}
Let a distribution satisfy proportionality. Thus $U_0(P_0) \geq 0.5$, 
and $U_1(P_1) \geq 0.5$. Thus $U_0(P_1) = 1 - U_0(P_0) \leq 0.5$. We can repeat 
the same reasoning for agent 1 to get $U_1(P_0) \leq 0.5$. Thus the utility of the 
other piece is at most the utility of the current piece. Thus this division is 
envy-free.

\paragraph*{Backward}
Let a distribution satisfy envy-freeness. Then $U_0(P_0) \geq U_0(P_1)$, and 
$U_1(P_1) \geq U_1(P_0)$. Since $P_0 + P_1 = 1$, $U_0(P_0) + U_0(P_1) = 1$. Then since 
it is envy free, we must have $U_0(P_0) \geq 0.5$. We can repeat the same reasoning for 
agent 1 to get $U_1(P_1) \geq 0.5$. Thus the division is proportional.

$\blacksquare$
\newpage


\section*{Problem 4}
\paragraph*{Guarantee}
Since Bob choose first, he will choose the piece that maximizes his utility. 
Whichever way Alice cuts the cake, there will be at least 1 piece with utility 
$U_B(P_i) \geq 1/3$, thus Bob will pick that one and will get at least 1/3 of their value of the cake.

If Alice divides the cake into sections such that $U_A(P_0) = U_A(P_1) = U_A(P_2)$, then it does not matter 
how Bob and Carlos picks. She will get at least 1/3 of the value of the cake.

\paragraph*{Envy-free}
This protocol is not envy-free. Consider the following example.

Alice values the first sections of the cake highly. Let $P_0 = [0, 1/9], P_1=[1/9, 2/9], P_2=[2/9, 1]$, and 
$U_A(P_i) = 1/3$. Let Bob and Carlos have a uniform utilities: $U_B(P_0) = 1/9, U_B(P_1) = 1/9, U_B(P_2) = 7/9$, $U_C = U_B$.

Then Bob will pick the third piece since it is the highest utility, and Carlos will pick the second or first piece. However, 
Carlos will envy Bob since according to $U_C$, $P_2$ is also the highest utility piece.

\newpage


\section*{Problem 5}
\paragraph*{No more agents left}
If we terminate the protocol with no more agents left, we give the remainder of the cake to the last agent. In this scenario, each agent got at least 1/3 of the cake 
per the protocol. Thus formaly we have $U_i(P_i) \geq 1/3$ for all $i \in [1, n]$. To violate 1/3-envy free, 
we must have $U_i(P_j) > 2/3$ for some $i, j \in [1, n]$. However, this is impossible since the sum of the utilities is 
1 and $U_i(P_i) + U_i(P_j) > 1$ by our protocol. Thus such a $j$ cannot exist, and the protocol is 1/3-envy free.

\paragraph*{No more cake left}
If we terminate the protocol with no more cake left, then we have $U_i(P_i) \geq 1/3$ for all $i \in [1, m]$, let the 
random agent who got a piece be $m+1$, since they did not call out, we have $0<U_{m+1}P(m+1)<1/3$, and $U_i(P_i) = 0$ for all $i \in [m+2, n]$.
In this scenario, if $U_i(P_j)>1/3$ for $i \in [m+1, n], j \in [1, m]$, they would have called out and gotten a piece, thus 
we guarantee that the difference in utility is at most 1/3, and the protocol is 1/3-envy free.

$\blacksquare$
\newpage


\section*{Problem 6}
\subsection*{1}
If all agents have the same utility. $U_A = U_B = U_C$, then the cake is cut into 
$P_0 = P_1 = 1/2$ by Alice. Bob and Alice will cut it into pieces that are equal in value 
to all 3 of them, resulting in each piece having utility of 1/6. Each agent then picks 2 pieces, 
and gets a utility of 1/3. Thus the partition satisfies proportionality.

\subsection*{2}
The cake is cut into 6 pieces. Suppose that we cannot find two pieces that has a sum that is greater in utility than 1/3. 
Then we observe the greatest $U(P_i + P_j)$ for any $i, j$ is less than 1/3. Thus $\sum_i P_i < 3 \times 1/3 = 1$. 
However this is a contradiction since the sum of the whole cake is 1. Thus we can always find two pieces that has a sum that is 
at least 1/3.

Since Carlos picks first, he can always guarantee a utility of at least 1/3.

\subsection*{3}
Alice first cuts the piece of cake into 2 pieces that are equal good to her and 
Bob. Then Bob and Alice both cut a piece of cake into 3 pieces that are equally good for each other. 
By the previous reasoning, Carlos can always guarantee a utility of at least 1/3. Since all the pieces are 
cut to be equal in value to Alice and Bob, they are each guaranteed to hae a utility of at least 1/3.
Thus we satisfy proportionality.

\subsection*{Extra Credit}
\paragraph*{Split into 2}
Define a function $f = U_0(P_0) - U_1(P_1)$. Since the cake cutting algorithm is essentially integrating a PDF, this 
function is continuous. Then since we are cutting it into two, $P_1 = 1-P_0$, so $f= U_0(P_0) - U_1(1-P_0)$. Suppose we 
define $x$ to be the location of the cut, so $P_0 = [0, x], P_1=[x,1]$. Then we have $f(0)=-1$ since agent 1 gets the whole cake, 
and $f(1)=1$ since agent 0 gets the whole cake. By the intermediate value theorem, there must exist a $c \in (0, 1)$ such that $f(c)=0$.
This means that there must exist a cut such that the two pieces are equal in value to both agents.

\paragraph*{Split into 3}
Define a function $f = U_0(P_0) - U_0(P_2) + U_1(P_0)-U_1(P_1)$. Let $P_0 = [0, x], P_1=[x, y], P_2=[y, 1]$. Then we have 
$f(0, 0) = -1$ since agent 1 gets the whole cake, and $f(1, 1) = 2$ since agent 0 gets the whole cake. 
$f(0, 1) = -1$
By the intermediate value theorem, there exists $(x, y)$ such that $f(x,y)=0$, and thus we can find a cut such that the three pieces 
are equal in value to both agents.

\newpage

\section*{Problem 7}
We can analyze this through different cases.

\subsection*{Two best pieces are equally good to Bob}
\paragraph*{Carina}
Since Carina choose first, she will choose the piece that is best for her.

\paragraph*{Bob}
Then Bob will choose the piece that is best for him, we know that 
this exists since the two most valuable pieces are equally good to Bob. 

\paragraph*{Alice}
Finally, Alice will choose the remaining piece. Since Alice cut the cake into 3 pieces that are equally good to her, 
she does not envy Bob or Carina. 

Thus the protocol is envy-free.

\subsection*{Carina does not picks $Q_1$}
\paragraph*{Bob}
If Carina does not pick $Q_1$, then Bob will pick $Q_1$ per the protocol. Of $Q_1, Q_2, Q_3$, Bob values 
$Q_1, Q_2$ equally, so he does not envy the agent who gets a piece from $Q$. For the allocation of $P$, since 
Bob picks first, he does not envy any other agent.

\paragraph*{Carina}
Since Carina picks first out of $Q_1, Q_2, Q_3$, she does not envy any other agent for the $Q$ selection. For $P$, 
since she divided it into 3 pieces that are equally good to her, she does not envy any other agent. 

\paragraph*{Alice}
For Alice, she divided $Q_1+P, Q_2, Q_3$, since $Q_1$ is chosen by Bob, she gets the untrimmed $Q_2$ or $Q_3$, and thus 
does not envy any other agent for the $Q$ pieces. Since Bob trimmed $Q_1$, Bob's share is $Q_1 + P_1$, and since Alice 
regards $Q_1+P = Q_2 = Q_3$, she would not envy Bob since $P_1 < P$. Since she is picking before Carina, she gets to choose between 
$Q_2+P_2$ and $Q_2+P_3$ or between $Q_3+P_2$ and $Q_3+P_3$, and since she considers $Q_2 = Q_3$, she will pick the $P$ piece 
that is most valuable for her, and thus does not envy Carina.

Thus the protocol is envy-free.

\subsection*{Carina picks $Q_1$}
\paragraph*{Bob}
If Carina picks $Q_1$, then Bob will pick $Q_2$ since now $U_B(Q_1) = U_B(Q_2) \geq U_B(Q_3)$. 
Since $Q_2, Q_3$ are equally good to Bob, he does not envy Carina for the $Q$ piece. For the $P$ piece, 
Bob cuts it into 3 pieces that are equally good to him, and thus does not envy Carina or Alice.

\paragraph*{Carina}
Since Carina picks first out of $Q_1, Q_2, Q_3$, she does not envy any other agent for the $Q$ selection. For $P$, 
since she is picking first, she does not envy any other agent for the $P$ piece. 

\paragraph*{Alice}
We can repeat the same reasoning as in the previous case. Alice divided $Q_1+P, Q_2, Q_3$, since $Q_1$ is chosen by Carina, she gets the untrimmed $Q_2$ or $Q_3$, and thus 
does not envy any other agent for the $Q$ pieces. Since Bob trimmed $Q_1$, Carina's share is $Q_1 + P_1$, and since Alice 
regards $Q_1+P = Q_2 = Q_3$, she would not envy Carina since $P_1 < P$. Since she is picking before Bob, she gets to choose between 
$Q_2+P_2$ and $Q_2+P_3$ or between $Q_3+P_2$ and $Q_3+P_3$, and since she considers $Q_2 = Q_3$, she will pick the $P$ piece 
that is most valuable for her, and thus does not envy Bob.

Thus the protocol is envy-free.
\end{document}